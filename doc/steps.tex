\documentclass[paper=a4, fontsize=11pt]{scrartcl}
\usepackage[T1]{fontenc} % Use 8-bit encoding that has 256 glyphs
\usepackage{fourier} % Use the Adobe Utopia font for the document - comment this line to return to the LaTeX default
\usepackage[english]{babel} % English language/hyphenation
\usepackage{amsmath,amsfonts,amsthm} % Math packages
\usepackage{sectsty} % Allows customizing section commands
\allsectionsfont{\centering \normalfont\scshape} % Make all sections centered, the default font and small caps
\usepackage{hyperref}


%----------------------------------------------------------------------------------------
%	TITLE SECTION
%----------------------------------------------------------------------------------------

\newcommand{\horrule}[1]{\rule{\linewidth}{#1}} % Create horizontal rule command with 1 argument of height

\title{
\normalfont \normalsize
\textsc{Arquillian Workshop} \\ [25pt] % Your university, school and/or department name(s)
\horrule{0.5pt} \\[0.4cm] % Thin top horizontal rule
\huge Testing JEE with Arquillian\\ % The assignment title
\horrule{2pt} \\[0.5cm] % Thick bottom horizontal rule
}

\author{Sebastiaan Heijmann} % Your name

\begin{document}

\maketitle % Print the title

\tableofcontents

\section{Prequisites}
For this workshop you need the following tools:
\begin{itemize}
		\item IDE (Netbeans preferred)
		\item WildFly application server
		\item Git
\end{itemize}

\section{Introduction}
Once you have installed all the prequisites and cloned the project from Github
you can see that there are 6 branches created (git branch -a). Every step in
this tutorial has an uncompleted and completed branch so you can check if the
implementation you wrote resembles the solution given.
\par
For further information about the Arquillian framework you can visit the
following links:
\begin{itemize}
	\item \url{http://arquillian.org/guides/getting_started/}
	\item \url{https://github.com/arquillian/arquillian-core}
\end{itemize}
\par
The project itself consists of 4 modules namely the parent, an EAR module, an EJB
module and a WAR module.
\begin{itemize}
		\item parent - responsible for managing shared dependencies.
		\item EAR - used to package and deploy the application.
		\item EJB - contains our JavaBeans.
		\item WAR - the web module containing the client code.
\end{itemize}
We will be writing tests for the EJB and WAR module so if you are ready to go do
a "git checkout -f STEP1" to begin implementing your first Arquillian test.

\section{Basic Arquillian test}
At this point we have a very simple class named UserRegistrator which does not
much but for the first test it is sufficient. The testclass skeleton is already
created and now it is up to you to further implement this test.
\par
Every Arquillian test needs a couple things namely:
\begin{itemize}
	\item Instruct the testclass to use the Arquillian JUnitRunner
		(\url{http://junit.sourceforge.net/javadoc/org/junit/runner/RunWith.html})
	\item Implement a static createDeployment method and annotate it with
		@Deployment. (this method creates your JavaArchive which is deployed in the
		container)
	\item Any number of test cases annotated with @Test
\end{itemize}

The createDeployment method is better to understand if you know what it should do
(bundling the resources our tests need so it can be deployed in WildFly).
For this test it is sufficient if we create a JavaArchive using ShrinkWrap's create
method which needs a parameter of the type of archive we want to create in this case
a JavaArchive (JAR).
\par
To this archive we can now add all our resources we will need for the tests. In
this case we need to add our class which we are going to test and a
ManifestResource with: a source (an instance of EmptyAsset), a target (beans.xml)
\par
We can give in a EmptyAsset.instance because JPA3 does noet care about the beans.xml
so it can be empty.
\par
If you managed to implement the two tests and want to compare the result you can do
a "git checkout -f STEP1\_solution".

\section{Integration test}
Please perform a "git checkout -f STEP2" to start the next use case.
\par
The project now has some more beans namely a User and a UserRepository. We will
leave the implementation of the UserRegistrator as it is and continue with a real
integration test between our datasource, our User entity and the UserRepository.
\par
The UserRepositoryIT class is the integration test which tests if JPA and our
repository is correctly working. For this we need a different deployment than in
the previous assignment because we are now using some JEE goodies which need a
WebArchive to be deployed from.
\par
Add the following resources to the Archive:
\begin{itemize}
	\item Add a normal resource with source META-INF/test-persistence.xml and
		target META-INF/persistence.xml
  \item Add an empty webinf resource named beans.xml
\end{itemize}
When you have succesfully created the Archive start implementing the given test
methods. If you're encountering problems with creating the archive you can use
its toString method to print the contents to the console.
\par
If you managed to implement the tests and want to compare the result you can do
a "git checkout -f STEP2\_solution".

\section{Functional test}
Please perform a "git checkout -f STEP3" to start the next use case.
\par
The previous assignments where all about testing inside the container, testing on
the server. The following test will be completely testing the client code. This
is possible because Arquillian can run in 3 modes as explained in the presentation.
\par
In our WAR module we now have 2 JSF pages, a UserController and a testclass. The
pages are as simplistic as possible which makes it easier to understand the
testcase.
\par
What the views do is the following:
\begin{itemize}
	\item You visit the register page
	\item Fill in you username
	\item Click on register
	\item You are Redirected to the home page
	\item Which displays a message to you.
\end{itemize}
In our testcase we capture the elements we need from the DOM using WebElement
instances and we use a WebDriver to browse throught pages.
\par
Important is again the deployment of your archive which must contain all the
required resources to be able to deploy the application. Therefor perform the
following steps:
\begin{itemize}
	\item Add a normal resource with source META-INF/test-persistence.xml and
		target META-INF/persistence.xml
	\item add a resource empty beans.xml
	\item add the web pages as a web resource create 2 new files named:\\
		- WEBAPP\_SRC + "home.xhtml"\\
		- WEBAPP\_SRC + "register.xhtml"\\
		Name the respectively "home.xhtml" and "register.xhtml".
	\item add faces config by adding a webinf resource with source "new StringAsset("<faces-config version=\"2.0\"/>")"
		 and target: faces-config.xml
	\item Make the deployment method: testable = false
\end{itemize}

We are now ready to inject the browser and WebElements we need into our testcase.
For this you will need to:
\begin{itemize}
		\item  Inject a WebDriver by annotating the WebDriver with @Drone
		\item Inject a WebElement for the inputfield by annotating the element
			with @FindBy.
		\item Do the same for the registration button.
		\item And the same for the JSF message bar which can be found by the
			tagname: "li" instead of finding by css.
\end{itemize}

If everythin went ok we now have a browser from which we can browse from
the deploymentUrl to an externalForm named "register.jsf"
\par
From here we can input a username and click on the button and we should now see
a message which is displayed in our JSF message bar.
\par
If you managed to implement the two tests and want to compare the result you can do
a "git checkout -f STEP3\_solution".\\

\textbf{Congratulations with finishing this workshop} Hopefully you have gained a
basic knowledge about Arquillian and what is possible with it.

\end{document}
